%%%%%%%%%%%%%%%%%%%%%%% file template.tex %%%%%%%%%%%%%%%%%%%%%%%%%
%
% This is a template file for Web of Conferences Journal
%
% Copy it to a new file with a new name and use it as the basis
% for your article
%
%%%%%%%%%%%%%%%%%%%%%%%%%% EDP Science %%%%%%%%%%%%%%%%%%%%%%%%%%%%
%

\documentclass[]{webofc}
%%% "twocolumn" for typesetting an article in two columns format (default one column)
%%%\documentclass{webofc}

\usepackage[varg]{txfonts}   % Web of Conferences font
%
% Put here some packages required or/and some personal commands
%
%
\begin{document}
%
\title{Building a Global HEP Software Training Community}
%
% subtitle is optional
%
%%%\subtitle{Do you have a subtitle?\\ If so, write it here}

\author{
    \firstname{First author} \lastname{First author}\inst{1,3}\fnsep\thanks{\email{Mail address for first author}} 
    \and
    \firstname{Second author} \lastname{Second author}\inst{2}\fnsep\thanks{\email{Mail address for second author if necessary}} \and
    \firstname{Third author} \lastname{Third author}\inst{3}\fnsep\thanks{\email{Mail address for last  author if necessary}}
    % etc
}

\institute{Insert the first address here
\and
           the second here
\and
           Last address
          }

\abstract{%
  To meet the computing challenges of upcoming experiments, software training efforts play an essential role in imparting best practices and popularizing new technologies. Because many of the taught skills are experiment-independent, the HSF/IRIS-HEP training group coordinates between different training initiatives while building a training center that provides students with various training modules. Both the events and the development of the training material are driven by a community of motivated educators. In this talk, we describe tools and organizational aspects with which we cultivate a strong sense of community ownership, provide recognition for individual contributions, and continue to motivate our members. We also describe new initiatives to foster further growth and increased reach. Among these is the evolution of our Training Center into a dynamic web page that allows us to significantly increase the scope of listed content without sacrificing readability.
}
%
\maketitle
%
% SLIDES: https://indico.jlab.org/event/459/contributions/11685/attachments/9665/14144/230509_hsf_training_community.pdf
%
\section{Introduction}
\label{intro}
Upcoming high-energy physics experiments such as the High Luminosity Large Hadron Collider (HL-LHC), the Long-Baseline Neutrino Facility (LBNF), and the Deep Underground Neutrino Experiment (DUNE) promise to yield an unprecedented volume of data.
However, the successful extraction of valuable insights from these experiments hinges upon effective data collection and analysis pipelines, a task in which software plays a pivotal role.

In this context, it is crucial to acknowledge that the majority of participants engaged in these large-scale collaborations have been primarily trained as physicists rather than software engineers. Consequently, the challenge of harnessing the full physics potential of the acquired data is compounded by the limited software engineering expertise within these collaborations. Moreover, the ever-evolving landscape of software technologies and data science methodologies poses an additional obstacle, as staying current with the latest advancements demands an accelerated adaptation process.

Addressing this challenge requires a strategic approach that prioritizes the training of physicists in all software-related skills. To fully realize the potential of current and next-generation experiments, it is imperative that researchers receive comprehensive training to effectively navigate the software-driven aspects of data collection and analysis. This multifaceted training must encompass not only basic proficiency in one or more programming languages but also the cultivation of best coding practices, adept version control and reproducibility strategies, as well as advanced data analysis techniques, and machine learning.
%
\begin{thebibliography}{}
%
% and use \bibitem to create references.
%
\bibitem{RefJ}
% Format for Journal Reference
Journal Author, Journal \textbf{Volume}, page numbers (year)
% Format for books
\bibitem{RefB}
Book Author, \textit{Book title} (Publisher, place, year) page numbers
% etc
\end{thebibliography}

\end{document}

% end of file template.tex

<div id='footer'><table width='100%'><tr><td class='right'><a href='http://fusioninventory.org/'><span class='copyright'>FusionInventory 9.1+1.0 | copyleft <img src='/glpi/plugins/fusioninventory/pics/copyleft.png'/>  2010-2016 by FusionInventory Team</span></a></td></tr></table></div>
